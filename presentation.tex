\documentclass[10pt,ngerman]{beamer}

\usepackage[ngerman]{babel}

\usepackage{csquotes}

\usetheme[progressbar=frametitle]{metropolis}
\usepackage{appendixnumberbeamer}

\usepackage[utf8]{inputenc}

\usepackage[bibencoding=utf8,
			sortlocale=de,
			style=numeric,
			pagetracker=true,
			autocite=inline,
			backrefstyle=three+,
			date=short,
			sorting=nty,
			backend=biber]{biblatex}
\addbibresource{Literaturverzeichnis.bib}
\renewcommand\bibname{Literaturverzeichnis}

\usepackage{booktabs}
\usepackage[scale=2]{ccicons}

\usepackage{xcolor, soul}
\definecolor{codebackground}{rgb}{0.95, 0.95, 0.92}
\definecolor{Black}{rgb}{0, 0, 0}


\usepackage{graphicx}

\usepackage{siunitx}
\sisetup{
  locale = DE ,
  detect-all,
  binary-units = true
}

\usepackage{listings}
\lstdefinestyle{MyLatexStyle} {
  frame=tb, % hrule above and below
  keepspaces=true,
  breaklines=true,
  columns=flexible,
  basicstyle=\tt\scriptsize,
  escapeinside={(*@}{@*)}, 
  backgroundcolor=\color{codebackground},
  showstringspaces=false,% for escapin
  language=[LaTeX]TeX,
  keywordstyle=\color{blue},
  identifierstyle=\color{magenta},
  stringstyle=\color{red},
  commentstyle=\color{teal},
  gobble=4
}
\lstdefinestyle{MyPythonStyle}{
  frame=tb, % hrule above and below
  keepspaces=true,
  breaklines=true,
  columns=flexible,
  basicstyle=\tt\scriptsize,
  escapeinside={(*@}{@*)}, % for escaping
  backgroundcolor=\color{codebackground},
  showstringspaces=false,
  language=Python,
  keywordstyle=\color{blue},
  stringstyle=\color{red},
  commentstyle=\color{teal},
  numbers=left, % {none, left, right}
  firstnumber=1,
  numberstyle=\scriptsize\color{black},
  numbersep=5pt,
  xleftmargin=5.0ex,
  gobble=4
}

\lstset{literate=%
    {Ö}{{\"O}}1
    {Ä}{{\"A}}1
    {Ü}{{\"U}}1
    {ß}{{\ss}}1
    {ü}{{\"u}}1
    {ä}{{\"a}}1
    {ö}{{\"o}}1
    {~}{{\textasciitilde}}1
}

\usepackage{caption}

%%% Für Quotes %%%
\usepackage{url}
\usepackage[ngerman]{varioref}
\usepackage{hyperref}
\setlength{\parindent}{0em}
\usepackage{cleveref}
\crefname{paragraph}{Abschnitt}{Abschnitt}

\usepackage{xspace}
\newcommand{\themename}{\textbf{\textsc{metropolis}}\xspace}
\definecolor{mSybilaRed}{HTML}{990000}

\setbeamercolor{title separator}{
  fg=mSybilaRed
}

\setbeamercolor{progress bar}{%
  fg=mSybilaRed,
  bg=mSybilaRed!90!black!30
}

\setbeamercolor{progress bar in section page}{
  use=progress bar,
  parent=progress bar
}

\setbeamercolor{alerted text}{%
  fg=mSybilaRed
}

\title{Projektpräsentation}

% \titlegraphic{\hfill\includegraphics[height=2.5cm]{pictures/python-logo.png}}

%\titlegraphic{\hfill\includegraphics[height=2.5cm]{pictures/LaTeX_logo.png}}
%\titlegraphic{\hfill\includegraphics[height=0.6cm]{sybila-logo/new.png}}
%\titlegraphic{\hfill\includegraphics[height=0.6cm]{sybila-logo/old.png}}
%\titlegraphic{\hfill\includegraphics[height=0.6cm]{sybila-logo/old-flat.png}}

\date{12.05.2022}
\author{Julius Caesar, Péter Egermann, Paul Görtler, Johannes Leyrer}
\institute{BSZ für Elektrotechnik Dresden -- IT20/2}

\subtitle{Die hard- und softwaretechnische Implementierung eines CO$_2$-Sensors zur Messung der Raumluftqualität}
%\institute{Center for modern beamer themes}
% \titlegraphic{\hfill\includegraphics[height=1.5cm]{logo.pdf}}

\setbeamertemplate{footline}
{
  \leavevmode
  \hbox{
  \begin{beamercolorbox}[wd=.15\paperwidth,ht=2.25ex,dp=1ex,center]{title in head/foot}
    \usebeamerfont{author in head/foot} % \insertshortauthor
  \end{beamercolorbox}

  \begin{beamercolorbox}[wd=.7\paperwidth,ht=2.25ex,dp=1ex,center]{author in head/foot}
    \usebeamerfont{author in head/foot}\insertshorttitle
  \end{beamercolorbox}

  \begin{beamercolorbox}[wd=.15\paperwidth,ht=2.25ex,dp=1ex,center]{title in head/foot}
    \insertframenumber{} / \inserttotalframenumber
  \end{beamercolorbox}
  }
}

\begin{document}

\maketitle

\begin{frame}{Gliederung}
    \setbeamertemplate{section in toc}[sections]
    \tableofcontents[hideallsubsections]
\end{frame}

\section{Einleitung}
\begin{frame}[fragile]{Einleitung}
    \begin{minipage}[t]{1\textwidth}
        \begin{quotation}
            Habt ihr bereits Erfahrungen mit CO$_2$-Sensoren gemacht?
        \end{quotation}
    \end{minipage}
\end{frame}

% \section{CO$_2$-Grenzwerte für eine unbedenkliche Atemluft}
% \begin{frame}[fragile]{CO$_2$-Grenzwerte für eine unbedenkliche Atemluft}
%   \begin{itemize}
%     \item Atmosphäre hat 400 ppm CO$_2$ \autocite{umweltbundesamt}
%     \item ab 1000 ppm CO$_2$ bedenklich laut DGUV ASR A3.6 \autocite{ASR}
%     \item ab 950 ppm CO$_2$ bedenklich laut DIN EN 16798-1 \autocite{din_en_16798}
%   \end{itemize}
% \end{frame}

% \begin{frame}[fragile]{CO$_2$-Grenzwerte für eine unbedenkliche Atemluft}
%   \begin{table}
%     \caption{nach DGUV ASR A3.6 \autocite{ASR}}
%     \begin{tabular}{ |c|p{0.49\textwidth}|}
%       \hline
%       CO$_2$-Konzentration in ppm & Bewertung               \\ \hline
%       $<$1000                     & hygienisch unbedenklich \\ \hline
%       1000-2000                   & hygienisch auffällig    \\ \hline
%       $>$2000                     & hygienisch inakzeptabel \\ \hline
%     \end{tabular}
%   \end{table}
%   \begin{table}
%     \caption{nach DIN EN 16798-1 \autocite{din_en_16798}}
%     \begin{tabular}{|c|p{0.49\textwidth}|}
%       \hline
%       CO$_2$-Konzentration in ppm & Bewertung                 \\ \hline
%       $<$950                      & Hohe Raumluftqualität     \\ \hline
%       950-1200                    & Mittlere Raumluftqualität \\ \hline
%       1200-1750                   & Mäßige Raumluftqualität   \\ \hline
%       $>$1750                     & Niedrige Raumluftqualität \\ \hline
%     \end{tabular}
%   \end{table}
% \end{frame}

% \section{Auswirkungen eines zu hohen CO$_2$-Gehaltes in der Raumluft}
% \begin{frame}[fragile]{Auswirkungen eines zu hohen CO$_2$-Gehaltes in der Raumluft}
%   \begin{itemize}
%     \item verringerte Konzentrationsfähigkeit
%     \item verringerte Leistungsfähigkeit
%     \item Halsschmerzen
%     \item Kopfschmerzen
%     \item Unwohlsein
%     \item Müdigkeit
%     \item Hustenanfälle
%   \end{itemize}

%   Quellen: \autocite{umweltbundesamt} \autocite{din_en_16798} \autocite{ASR} \autocite{kajtar} \autocite{zhang} \autocite{myhrvold} \autocite{tiesler}
% \end{frame}

% \section{Hardwaretechnische Umsetzung}

% \begin{frame}[fragile]{Raspberry Pi 3B+}
%   \begin{figure}
%     \centering
%     \captionsetup{justification=centering}
%     \includegraphics[width=0.55\textwidth]{pictures/RasPi.png}
%     \caption{Raspberry Pi 3B+ \autocite{rasPi}}
%   \end{figure}
% \end{frame}

% \begin{frame}[fragile]{CO$_2$-Sensor}
%   \begin{figure}
%     \centering
%     \captionsetup{justification=centering}
%     \includegraphics[trim={0 10cm 0 10cm},clip,width=0.85\textwidth]{pictures/co2monitor.png}
%     \caption{TFA Dostmann AIRCO2NTROL MINI \autocite{co2monitor}}
%   \end{figure}
% \end{frame}


% \section{Softwaretechnische Umsetzung}

% \subsection{Grundlagen}
% \begin{frame}[fragile]{Grundlagen}
%   \begin{minipage}[t]{0.49\textwidth}
%     \begin{itemize}
%       \item Linux-Distribution inklusive mitgelieferter Standardsoftware
%       \item Docker
%       \item Python
%       \item FastAPI
%       \item React
%       \item ChartJs
%       \item SQLite
%     \end{itemize}
%   \end{minipage}
%   \begin{minipage}[t]{0.49\textwidth}
%     \begin{figure}
%       \centering
%       \captionsetup{justification=centering}
%       \includegraphics[width=1\textwidth]{pictures/SoftwareKomponenten.png}
%       \caption{Verwendete Softwarekomponenten \autocite{dockerLogo}\autocite{fastapiLogo}\autocite{sqliteLogo}\autocite{reactLogo}\autocite{pythonLogo}\autocite{chartjsLogo}}
%     \end{figure}
%   \end{minipage}
% \end{frame}

% \subsection{Zusammenspiel der Softwarekomponenten}

% \begin{frame}[fragile]{Zusammenspiel der Softwarekomponenten}
%   \begin{figure}
%     \centering
%     \captionsetup{justification=centering}
%     \includegraphics[width=0.6\textwidth]{pictures/SoftwareZusammenspiel.png}
%     \caption{Zusammenspiel der Softwarekomponenten \autocite{co2monitor}\autocite{rasPi}}
%   \end{figure}
% \end{frame}

% \subsection{Aufbau und Einrichtung der Softwarekomponenten}

% \begin{frame}[fragile]{Aufbau und Einrichtung der Softwarekomponenten}

%   \begin{minipage}[t]{0.49\textwidth}
%     \begin{itemize}
%       \item Backend: Python mit FastAPI
%       \item Frontend: React und ChartsJs
%       \item Lese-Software: Python-Script
%       \item Datenbank: SQLite
%     \end{itemize}
%   \end{minipage}

%   \begin{lstlisting}[language=Bash]
%     docker-compose -f docker-compose.yml up -d
%   \end{lstlisting}
% \end{frame}

% \section{Fazit}
% \begin{frame}[fragile]{Fazit}
%   \begin{minipage}[t]{0.80\textwidth}
%     \textbf{Ergebnisse:}\newline
%     \begin{itemize}
%       \item bestätigte Relevanz der Raumluftqualität
%       \item bestätigte Verbindung zwischen hohen CO$_2$-Konzentrationen und verminderter Konzentrationsfähigkeit/Produktivität
%       \item schaffen einer kostengünstigen Möglichkeit zur selbstständigen Kontrolle der Raumluftqualität
%     \end{itemize}
%   \end{minipage}
% \end{frame}

\begin{frame}[standout]
    Fragen?
\end{frame}

\appendix

\begin{frame}[allowframebreaks]{Literaturverzeichnis}

    \printbibliography[title={Quellenverzeichnis}]

\end{frame}

\begin{frame}[standout]
    Danke für die Aufmerksamkeit!
\end{frame}
\end{document}
